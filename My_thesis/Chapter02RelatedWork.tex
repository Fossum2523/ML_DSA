%
\ifdefined\integrated\else
\documentclass[12pt, a4paper, fleqn]{./styles/ntust_report}
\usepackage{graphicx}
\usepackage{booktabs}
\usepackage{multirow}
\input{./styles/ntust_style_thesis.tex}
%\usepackage{wallpaper}
%\CenterWallPaper{1}{ntust_watermark.jpg}
\setcounter{page}{1}
\begin{document}
\baselineskip=24pt  % 24 lines per page
\setcounter{chapter}{1} \fi

\chapter{ML-DSA 演算法與技術背景}

本章將介紹 ML-DSA 整體演算法架構所依據的密碼學基礎與技術背景。內容安排如下:

\begin{itemize}
    \item \textbf{2.1 節} 說明後量子密碼學的發展背景與 NIST PQC 標準簡介,並介紹 ML-DSA 所依據的 Module-LWE 與 Module-SIS 數學問題與安全性來源。
    \item \textbf{2.2 節} 詳細說明 ML-DSA 的簽章演算法流程,包括金鑰生成、簽章與驗證,並解釋 ML-DSA-44 參數組的意涵與設計選擇。
    \item \textbf{2.3 節} 介紹實現中關鍵的數學運算:Number Theoretic Transform(NTT)與其反轉換(INTT),以及模數轉換、位元包裝等硬體化技巧。
    \item \textbf{2.4 節} 說明 SHAKE 雜湊函數於 ML-DSA 中的重要角色,包含 SHAKE128 與 SHAKE256 的運作與本研究對其硬體架構的設計方式。
\end{itemize}

%%%%%%%%%%%%%%%%%%%%%%%%%%%%%%%%%%%%%%%%%%%%%%%%%%%%%%%%%%%%%%%%%%%%%%%%%%%%%%%%%%%%%%%%%%%%%%%%%%%%%%%%%%%%%%%%%%%%%%%%%%%%%%%%%%%%%%%%%%%%%%%%%%%%
\section{後量子密碼學簡介}
\subsection{後量子密碼學背景與標準化}

隨著 Shor 演算法的問世,傳統以整數分解與離散對數問題為基礎的公鑰密碼系統(如 RSA、ECC)將無法抵禦量子電腦的攻擊。為因應此一挑戰,美國國家標準與技術研究院(NIST)於 2016 年啟動「後量子密碼學(Post-Quantum Cryptography, PQC)」標準制定計畫,歷經三輪評選與數十種候選演算法之競爭,最終於 2024 年公布 ML-DSA 作為數位簽章的標準之一。

ML-DSA(Module-Lattice-Based Digital Signature Algorithm)係以 CRYSTALS-Dilithium 為基礎,採用模組格(Module Lattice)問題作為安全性依據,結合高效的多項式計算與雜湊生成機制,具備短簽名長度、運算速度快、且硬體實作友善等特性。



\subsection{Module-LWE 問題(模組化學習誤差)}

模組化學習誤差問題(Module Learning With Errors, MLWE)為傳統 LWE 問題的模組格推廣形式,其核心思想為:在高維模組格中加入小誤差後,使得線性關係難以識別。

\textbf{定義:} 給定公開參數 $A \in R_q^{k \times l}$,秘密向量 $s_1 \in R_q^l, s_2 \in R_q^k$,計算公開項:
\[
A =
\begin{bmatrix}
a_{11} & a_{12} & \cdots & a_{1l} \\
a_{21} & a_{22} & \cdots & a_{2l} \\
\vdots & \vdots & \ddots & \vdots \\
a_{k1} & a_{k2} & \cdots & a_{kl}
\end{bmatrix}
\in R_q^{k \times l}, \quad
s_1 =
\begin{bmatrix}
s_{11} \\
s_{12} \\
\vdots \\
s_{1l}
\end{bmatrix}\quad
s_2 =
\begin{bmatrix}
s_{21} \\
s_{22} \\
\vdots \\
s_{2l}
\end{bmatrix}
\in R_q^l,
\quad
t = A \cdot s_1 + s_2 \in R_q^k
\]
攻擊者須判斷 $(A, t)$ 是否為上述結構或隨機生成的 $(A, t')$,此即為決定型 MLWE 問題。安全性來自於從 $(A, t)$ 無法有效還原 $s_1, s_2$,即使是量子演算法亦無解。

MLWE 廣泛應用於公開金鑰生成與抗選擇密文攻擊設計中,是 ML-DSA 提供「金鑰保密性」的核心安全假設。


\subsection{Module-SIS 問題(模組最短整數解問題)}
模組最短整數解問題(Module Short Integer Solution, MSIS)為格基密碼學中常見的假設之一。其目標為:在模 $q$ 多項式環中尋找非零短向量 $z$,使其映射後為零。

\textbf{定義:} 給定 $A \in R_q^{k \times l}$,尋找非零向量 $z$ 滿足:
\[
A =
\begin{bmatrix}
a_{11} & a_{12} & \cdots & a_{1l} \\
a_{21} & a_{22} & \cdots & a_{2l} \\
\vdots & \vdots & \ddots & \vdots \\
a_{k1} & a_{k2} & \cdots & a_{kl}
\end{bmatrix}
\in R_q^{k \times l}, \quad
z =
\begin{bmatrix}
z_1 \\
z_2 \\
\vdots \\
z_l
\end{bmatrix}
\in R_q^l, \quad
\text{such that} \quad
A \cdot z = 0 \mod q
\]


MSIS 通常用於證明數位簽章的「不可偽造性」,在 ML-DSA 中,與檢查簽章是否落入合法集合及誤差範圍有關。



\subsection{Fiat-Shamir with Aborts 架構}

Fiat-Shamir Heuristic 是將互動式證明轉為非互動數位簽章的經典技術,其流程包含:
\begin{enumerate}
  \item \textbf{承諾(Commitment)}:產生隨機向量,計算中間值傳送給驗證者。
  \item \textbf{挑戰(Challenge)}:由驗證者(或雜湊函數)產生隨機挑戰值。
  \item \textbf{響應(Response)}:簽章者依據挑戰值計算響應,驗證者可用以驗證正確性。
\end{enumerate}

為防止側信道洩漏,Fiat-Shamir with Aborts 架構進一步加入「拒絕抽樣」機制,即若響應值 $z$ 不符合某些範數條件,則重新抽樣,直到產生合格響應。

此機制能有效:
\begin{itemize}
  \item 避免 $z$ 與私鑰具統計關聯
  \item 提升安全性(防私鑰重建攻擊)
  \item 增加可實作性(支援均勻雜湊與簡單取樣)
\end{itemize}



\subsection{ML-DSA 簽章流程步驟}

ML-DSA 簽章流程採用 Fiat-Shamir with Aborts,具體步驟如下:

\begin{enumerate}
  \item \textbf{生成承諾值}:產生 $y \in R_q^l$,計算 $w = A \cdot y$ 並四捨五入得 $w_1$。
  \item \textbf{挑戰生成}:計算 $\mu = H(tr \| M)$ 並與 $w_1$ 雜湊產生 challenge $c$。
  \item \textbf{響應計算}:計算 $z = y + c \cdot s_1$。
  \item \textbf{提示值計算}:從 $z$ 與 $t_1$ 推出提示值 $h$,幫助驗證者重建 $w_1$。
  \item \textbf{組合簽章}:輸出簽章三元組 $\sigma = (\tilde{c}, z, h)$。
  \item \textbf{拒絕機制}:若 $z$ 或 $r_0$ 不符範圍,或 $h$ 太長,則拒絕並重試。
\end{enumerate}

此流程設計結合模組格安全性與實用性,使 ML-DSA 成為後量子環境下具備抗量子、效率佳的數位簽章演算法。
%%%%%%%%%%%%%%%%%%%%%%%%%%%%%%%%%%%%%%%%%%%%%%%%%%%%%%%%%%%%%%%%%%%%%%%%%%%%%%%%%%%%%%%%%%%%%%%%%%%%%%%%%%%%%%%%%%%%%%%%%%%%%%%%%%%%%%%%%%%%%%%%%%%%



%%%%%%%%%%%%%%%%%%%%%%%%%%%%%%%%%%%%%%%%%%%%%%%%%%%%%%%%%%%%%%%%%%%%%%%%%%%%%%%%%%%%%%%%%%%%%%%%%%%%%%%%%%%%%%%%%%%%%%%%%%%%%%%%%%%%%%%%%%%%%%%%%%%%
\section{ML-DSA 演算法流程與參數敘述}

本節將詳細說明 ML-DSA 簽章系統所使用的三大核心演算法,包括金鑰產生(KeyGen)、簽章產生(Sign)與簽章驗證(Verify),並列出各演算法的運作邏輯與參數配置方式,說明其如何根據不同安全等級設定簽章參數與誤差邊界,作為後續硬體實作與模組設計之依據。

\subsection{Key Generation(金鑰產生)}

金鑰產生演算法負責產生簽章系統的公開金鑰與私密金鑰,流程如下:

\begin{enumerate}
  \item 使用亂數種子 $\zeta \in \{0,1\}^{256}$ 經雜湊後生成 $(\rho, \rho', K)$ 三組種子。
  \item 根據 $\rho$ 以擴展雜湊函數 ExpandA 生成公開矩陣 $A \in R_q^{k \times l}$。
  \item 根據 $\rho'$ 生成秘密向量 $s_1 \in R_q^l$、$s_2 \in R_q^k$。
  \item 計算 $t = A \cdot s_1 + s_2$,並使用 Power2Round 將 $t$ 拆分為高位元 $t_1$ 與低位元 $t_0$。
  \item 生成公鑰摘要 $tr = H(\rho \| t_1)$。
  \item 組成公鑰 $pk = (\rho, t_1)$,私鑰 $sk = (\rho, K, tr, s_1, s_2, t_0)$。
\end{enumerate}

\begin{figure}[H]
    \centering
    \includegraphics[width=0.4\textwidth]{figures/Algorithm/KeyGen.png}
    \caption{ML-DSA 金鑰產生流程示意圖}
\end{figure}

\subsection{Signature Generation(簽章產生)}

簽章演算法負責對訊息 $M$ 進行加密簽章,流程如下:

\begin{enumerate}
  \item 由私鑰 $sk$ 解出 $\rho, K, tr, s_1, s_2, t_0$,重新生成矩陣 $A$。
  \item 將 $tr \| M$ 雜湊生成訊息摘要 $\mu$,計算隨機遮罩種子 $\rho'$。
  \item 進入 rejection sampling 迴圈,直到產出合法簽章:
  \begin{itemize}
    \item 生成遮罩向量 $y \in R_q^l$,計算 $w = Ay$ 並擷取高位元 $w_1$。
    \item 使用雜湊 $H(\mu \| w_1)$ 產生挑戰 $c$。
    \item 計算回應值 $z = y + cs_1$,並檢查其範圍。
    \item 計算 $r_0 = LowBits(w - cs_2)$,並確認是否落於安全範圍內。
    \item 若條件不符,則丟棄本次嘗試,重新抽樣。
  \end{itemize}
  \item 產生提示值 $h = MakeHint(-ct_0, w - cs_2 + ct_0)$。
  \item 組合簽章 $\sigma = (\tilde{c}, z, h)$。
\end{enumerate}

\begin{figure}[H]
    \centering
    \includegraphics[width=0.6\textwidth]{figures/Algorithm/SignGen.png}
    \caption{ML-DSA 簽章產生流程示意圖}
\end{figure}

\subsection{Signature Verification(簽章驗證)}

驗章演算法負責驗證簽章的合法性與訊息完整性,流程如下:

\begin{enumerate}
  \item 根據公鑰 $pk = (\rho, t_1)$ 重新生成矩陣 $A$。
  \item 雜湊 $H(H(\rho \| t_1) \| M)$ 產生訊息摘要 $\mu$。
  \item 根據簽章中的 $\tilde{c}$ 還原挑戰 $c$。
  \item 根據提示值 $h$ 及 $z$ 還原 $w_1'$,並重新雜湊與原挑戰比較。
  \item 若 $\|z\|_\infty < \gamma_1 - \beta$ 且提示值符合範圍,且雜湊值一致,則簽章有效。
\end{enumerate}

\begin{figure}[H]
    \centering
    \includegraphics[width=0.5\textwidth]{figures/Algorithm/SignVer.png}
    \caption{ML-DSA 簽章認證流程示意圖}
\end{figure}

\subsection{參數設定與 ML-DSA-44 組態}

ML-DSA-44 對應 NIST category 2 安全等級,其參數組如下:
\begin{itemize}
  \item $k=4$, $\ell=4$, $n=256$, $q=8380417$
  \item $\eta=2$, $\gamma_1=2^{17}$, $\gamma_2=261888$, $\beta=78$, $\omega=80$
\end{itemize}


透過適當調整參數(例如提升 $k, \ell$ 或減小 $\gamma_2$),可達成更高的安全等級;相反亦可作為低功耗系統之參數下修。

本研究所實作之硬體架構即針對 ML-DSA-44 進行最佳化設計,並支援 rejection loop 的重複嘗試與雜湊 / NTT 模組的高重用,於後續章節說明其實作細節與優化策略。

%%%%%%%%%%%%%%%%%%%%%%%%%%%%%%%%%%%%%%%%%%%%%%%%%%%%%%%%%%%%%%%%%%%%%%%%%%%%%%%%%%%%%%%%%%%%%%%%%%%%%%%%%%%%%%%%%%%%%%%%%%%%%%%%%%%%%%%%%%%%%%%%%%%%


%%%%%%%%%%%%%%%%%%%%%%%%%%%%%%%%%%%%%%%%%%%%%%%%%%%%%%%%%%%%%%%%%%%%%%%%%%%%%%%%%%%%%%%%%%%%%%%%%%%%%%%%%%%%%%%%%%%%%%%%%%%%%%%%%%%%%%%%%%%%%%%%%%%%
\section{數論轉換(NTT)之數學背景與硬體考量}
\subsection{數論轉換基本概念}
數論轉換(Number Theoretic Transform, NTT)是一種在模數 $q$ 下進行多項式運算加速的技術,與傅立葉變換(FFT)在複數域的性質相似。NTT 將多項式乘法由傳統的 $O(n^2)$ 時間複雜度,降至 $O(n \log n)$,顯著提升運算效率。

數學上,NTT 可定義為:對向量 $(a_0, a_1, \ldots, a_{n-1})$,使用原根 $\omega$ 計算轉換後的向量 $(A_0, A_1, \ldots, A_{n-1})$,其中:


\[
A_k = \sum_{j=0}^{n-1} a_j \omega^{jk} \mod q
\]



\subsection{正向與反向 NTT:FFT 結構化加速}

數論轉換與其反變換可以藉由快速傅立葉變換(FFT)技巧來加速,最典型的就是 Cooley-Tukey (CT) 演算法與 Gentleman-Sande (GS) 演算法,分別用於正向 NTT 與反向 INTT。以下為更詳細的數學推導:

\textbf{Cooley-Tukey(CT)快速NTT推導:}

將傳統 NTT 的總和根據指標的奇偶性拆分:
\[
\hat{a}_j = \sum_{i=0}^{n-1} \psi^{2ij+i} a_i \mod q = \sum_{i=0}^{n/2-1} \psi^{4ij+2i} a_{2i} + \psi^{2j+1} \sum_{i=0}^{n/2-1} \psi^{4ij+2i} a_{2i+1} \mod q
\]
由於 $\psi^{n} = -1$,透過對稱性,我們也可以推得:
\[
\hat{a}_{j+n/2} = \sum_{i=0}^{n/2-1} \psi^{4ij+2i} a_{2i} - \psi^{2j+1} \sum_{i=0}^{n/2-1} \psi^{4ij+2i} a_{2i+1} \mod q
\]
設:
\[
A_j = \sum_{i=0}^{n/2-1} \psi^{4ij+2i} a_{2i}, \quad B_j = \sum_{i=0}^{n/2-1} \psi^{4ij+2i} a_{2i+1}
\]
則:
\[
\hat{a}_j = A_j + \psi^{2j+1} B_j \mod q
\]
\[
\hat{a}_{j+n/2} = A_j - \psi^{2j+1} B_j \mod q
\]


\textbf{Gentleman-Sande(GS)快速INTT推導:}

將總和根據上下半部拆解,令 $\hat{a}_j$ 為輸入之 NTT 領域向量,INTT 推導如下:
\[
a_i = \sum_{j=0}^{n-1} \psi^{-(2i+1)j} \hat{a}_j \mod q = \psi^{-i} \left[ \sum_{j=0}^{n/2-1} \psi^{-2ij} \hat{a}_j + \sum_{j=0}^{n/2-1} \psi^{-2i(j + n/2)} \hat{a}_{j + n/2} \right] \mod q
\]
偶數項推導:
\[
a_{2i} = \psi^{-2i} \sum_{j=0}^{n/2-1} \left( \hat{a}_j + \hat{a}_{j+n/2} \right) \psi^{-4ij} \mod q
\]
奇數項推導:
\[
a_{2i+1} = \psi^{-2i} \sum_{j=0}^{n/2-1} \left( \hat{a}_j - \hat{a}_{j+n/2} \right) \psi^{-4ij} \mod q
\]
設:
\[
A_i = \sum_{j=0}^{n/2-1} \hat{a}_j \psi^{-4ij}, \quad B_i = \sum_{j=0}^{n/2-1} \hat{a}_{j+n/2} \psi^{-4ij}
\]
則可得:
\[
a_{2i} = (A_i + B_i) \psi^{-2i} \mod q
\]
\[
a_{2i+1} = (A_i - B_i) \psi^{-2i} \mod q
\]


\subsection{2.3.3 蝴蝶結構(Butterfly Unit)與運算流程圖}

為了實現 NTT/INTT 的加速運算,演算法通常會被拆解為重複執行的小型處理單元,其核心為蝴蝶結構(Butterfly Unit)。蝴蝶結構負責進行資料配對後的加減與旋轉因子乘法,是整體 NTT 基本的運算元件。

以下為 Cooley-Tukey 與 Gentleman-Sande 所對應的蝴蝶結構圖示:

\begin{figure}[H]
    \centering
    \includegraphics[width=0.5\textwidth]{figures/NTT/Cooley-Tukey.png}
    \caption{Cooley-Tukey (CT) 蝴蝶結構}
    \label{fig:ct-butterfly}
\end{figure}

\begin{figure}[H]
    \centering
    \includegraphics[width=0.5\textwidth]{figures/NTT/Gentleman-Sande.png}
    \caption{Gentleman-Sande (GS) 蝴蝶結構}
    \label{fig:gs-butterfly}
\end{figure}

此兩種蝴蝶操作模式分別對應於 NTT 與 INTT 的迭代運算方式,並可依照資料流與控制邏輯進行流水線化處理。


\subsection{負捲積與模數選擇}

在 ML-DSA 中,多項式運算定義在模 $x^n + 1$ 的環上,需使用負捲積(Negacyclic Convolution)。

為此,必須選擇滿足以下條件的模數 $q$:
\begin{itemize}
  \item 存在階數為 $2n$ 的原根 $\psi \in \mathbb{Z}_q$。
  \item $q \equiv 1 \mod 2n$。
\end{itemize}



\subsection{小結}

透過 NTT 技術,ML-DSA 能將多項式乘法運算顯著加速,為實現高效簽章演算法奠定基礎。本研究針對 ML-DSA-44 的參數組,選擇 MDC 流水線架構,實作低延遲且可資源重用的 NTT 加速器,支援 FPGA 與 ASIC 雙平台部署。

%%%%%%%%%%%%%%%%%%%%%%%%%%%%%%%%%%%%%%%%%%%%%%%%%%%%%%%%%%%%%%%%%%%%%%%%%%%%%%%%%%%%%%%%%%%%%%%%%%%%%%%%%%%%%%%%%%%%%%%%%%%%%%%%%%%%%%%%%%%%%%%%%%%%


%%%%%%%%%%%%%%%%%%%%%%%%%%%%%%%%%%%%%%%%%%%%%%%%%%%%%%%%%%%%%%%%%%%%%%%%%%%%%%%%%%%%%%%%%%%%%%%%%%%%%%%%%%%%%%%%%%%%%%%%%%%%%%%%%%%%%%%%%%%%%%%%%%%%
\section{SHAKE 與雜湊運算在 ML-DSA 中的角色}
\subsection{SHA-3 與 SHAKE:從傳統雜湊到彈性輸出雜湊的演進}
SHA-3 是由 NIST 於 2015 年通過的第三代安全雜湊標準,其核心演算法 Keccak 採用與 SHA-2 完全不同的架構設計。SHA-3 系列可分為兩大類:四種固定長度的雜湊函數(SHA3-224、SHA3-256、SHA3-384、SHA3-512)與兩種可擴展輸出函數(Extendable-Output Functions, XOF),即 SHAKE128 與 SHAKE256。

XOF 最大的特點是可產生任意長度的輸出,因此特別適合用於金鑰擴展(KDF)、雜湊隨機數產生(PRNG)、訊息雜湊與數位簽章等密碼學應用 \cite{FIPS202}。


\begin{table}[htbp]
\centering
\caption{SHA-3 家族分類與安全強度比較}
\resizebox{\textwidth}{!}{%
\begin{tabular}{|c|c|c|c|c|c|}
\hline
Function & SHA3 Primitive & Output Size & Collision & Preimage & 2nd Preimage \\
\hline
\multirow{4}{*}{Cryptographic Hash Function}
    & SHA3-224 & 224 & 112 & 224 & 112 \\
    & SHA3-256 & 256 & 128 & 256 & 128 \\
    & SHA3-384 & 384 & 192 & 384 & 192 \\
    & SHA3-512 & 512 & 256 & 512 & 256 \\
\hline
\multirow{2}{*}{Extendable-Output Function}
    & SHAKE128 & arbitrary & up to 128 & up to 128 & up to 128 \\
    & SHAKE256 & arbitrary & up to 256 & up to 256 & up to 256 \\
\hline
\end{tabular}
}
\end{table}




\subsection{Sponge Construction:Keccak 的基礎運作原理}

SHA-3 函數採用 \textbf{海綿結構(Sponge Construction)},其基本過程分為兩階段:

\begin{itemize}
    \item \textbf{吸收階段(Absorb Phase)}:將輸入訊息分塊為 $r$ 位元,逐塊與 state 進行 XOR 運算,並送入 permutation function。
    \item \textbf{擠壓階段(Squeeze Phase)}:從 state 前 $r$ 位元中取出輸出資料,如需更長輸出則持續迭代。
\end{itemize}

整體 state 長度為 $b = r + c$,其中:
\begin{itemize}
    \item $r$:bitrate,每輪可處理輸入/輸出位元數,影響處理速率。
    \item $c$:capacity,用於提供碰撞抗性與原像安全性。
    \item SHA-3 固定採用 $b = 1600$。
\end{itemize}

\begin{figure}[H]
    \centering
    \includegraphics[width=1\textwidth]{figures/Keccak/sponge-construction.png}
    \caption{Sponge Construction 吸收與擠壓階段示意圖}
\end{figure}

SHA-3 採用 \textbf{pad10*1} 多速率填補規則以確保完整分塊,並使用 KECCAK-p[1600,24] 作為 permutation 核心 \cite{FIPS202}。




%\subsection{SHAKE 與 SHA-3 雜湊的差異與安全性考量}
%
%SHAKE128 與 SHAKE256 與 SHA3-256 等固定長度雜湊函數共用相同 sponge 結構與 permutation 核心,但輸出長度可由使用者自由指定,實現極高彈性。
%
%其安全強度如下所示 \cite{FIPS202}:
%
%\begin{table}[H]
%\centering
%\caption{SHAKE 雜湊函數的安全強度}
%\begin{tabular}{|c|c|c|c|}
%\hline
%\textbf{函數} & \textbf{Collision} & \textbf{Preimage} & \textbf{2nd Preimage} \\
%\hline
%SHAKE128 & $\min(d/2, 128)$ & $\min(d, 128)$ & $\min(d, 128)$ \\
%SHAKE256 & $\min(d/2, 256)$ & $\min(d, 256)$ & $\min(d, 256)$ \\
%\hline
%\end{tabular}
%\end{table}
%
%然而,SHAKE 的「可延展性」特性也可能帶來潛在風險:相同輸入的不同長度雜湊輸出具有重疊性,需仰賴 \textbf{domain separation(輸入尾碼)} 技術避免被利用。例如:
%\[
%\text{SHAKE128}(M, 224) = \text{Trunc}_{224}(\text{SHAKE128}(M, 512))
%\]




\subsection{SHAKE128 與 SHAKE256 在 ML-DSA 中的應用與硬體設計策略}

本研究只使用SHA-3中的SHAKE128 與 SHAKE256 兩種雜湊函數,皆為 SHA-3 標準中的可擴展輸出函數(Extendable-Output Functions, XOF)。其核心皆採用 KECCAK-p[1600,24] permutation,並透過 sponge 結構運作。與固定長度輸出的 SHA3-256 不同,SHAKE 提供任意長度的輸出能力,並透過輸入尾碼(例如 0x1F)實現 \textbf{domain separation},以避免不同應用間產生重疊輸出。

在 ML-DSA 的簽章與驗證流程中,SHAKE 的角色至關重要:
\begin{itemize}
    \item \textbf{SHAKE128}:用於產生公鑰矩陣 $A$(ExpandA)。此模組需高速率輸出,故使用 $r=1344$ 的 SHAKE128。
    \item \textbf{SHAKE256}:應用於大多數雜湊與亂數產生操作,如 ExpandS、ExpandMask、SampleInBall,以及訊息摘要雜湊(如 $H(tr\|M)$、$H(K\|rnd\|\mu)$ 等)。
\end{itemize}

下表整理了 ML-DSA 使用 SHAKE 的典型應用與資料規模:

\begin{table}[H]
\centering
\caption{SHAKE 應用於 ML-DSA 各模組之 Input/Output 資訊}
\begin{tabular}{|c|c|c|c|c|}
\hline
\textbf{模組} & \textbf{SHAKE 類型} & \textbf{應用階段} & \textbf{輸入(bits)} & \textbf{輸出(bits)} \\
\hline
ExpandA & SHAKE128 & KeyGen / Sign & 272 & 7152 \\
ExpandS & SHAKE256 & KeyGen & 528 & 3848 \\
ExpandMask & SHAKE256 & Sign & 528 & 4608 \\
SampleInBall & SHAKE256 & Sign / Verify & 256 & 1768 \\
$H(tr\|M)$ 等 & SHAKE256 & Sign / Verify & 多達 10496 & 512 \\
\hline
\end{tabular}
\end{table}

與現有文獻中常見的「多 Keccak 雜湊核心並行架構」相比,本研究以硬體資源效率為導向,採用僅 \textbf{單一 SHAKE 雜湊模組},並支援 \textbf{動態切換 SHAKE128/SHAKE256 模式}。透過 mode select 控制信號與動態參數設定,該模組可根據不同雜湊需求調整 $r$ 與 $c$ 值,並自動完成輸出長度的擠壓與輸入尾碼的配置,實現高度模組共用性。

此外,本設計亦加入以下結構以提升模組整體運作效率:
\begin{itemize}
    \item \textbf{Keccak 控制排程器(SHAKE controller)}:負責調度不同操作間的工作時序。
    \item \textbf{Hash input multiplexer}:支援不同資料來源之雜湊操作資料交錯。
    \item \textbf{輸出隊列與 pipeline 寄存器}:減少等待時間並提升模組利用率。
\end{itemize}
%%%%%%%%%%%%%%%%%%%%%%%%%%%%%%%%%%%%%%%%%%%%%%%%%%%%%%%%%%%%%%%%%%%%%%%%%%%%%%%%%%%%%%%%%%%%%%%%%%%%%%%%%%%%%%%%%%%%%%%%%%%%%%%%%%%%%%%%%%%%%%%%%%%%









\ifdefined\integrated\else
\input{backpages/ntust_backpages.tex}
\end{document} \fi
