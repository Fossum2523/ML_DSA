%
\ifdefined\integrated\else
\documentclass[12pt, a4paper]{./styles/ntust_report}
\input{./styles/ntust_style_thesis.tex}
%\usepackage{wallpaper}
%\CenterWallPaper{1}{ntust_watermark.jpg}
\setcounter{page}{1}
\begin{document}
\baselineskip=24pt  % 24 lines per page
\setcounter{chapter}{0} \fi

\chapter{緒論} \label{cha:intro}

\section{研究動機與方向}

隨著量子電腦技術逐步發展,傳統公鑰密碼演算法如 RSA 與 ECC 面臨潛在被破解的風險。為應對此威脅,美國國家標準與技術研究院(NIST)於 2016 年展開後量子密碼學(Post-Quantum Cryptography, PQC)標準制定,並於 2024 年正式發布 ML-DSA(Module-Lattice-Based Digital Signature Algorithm)作為新一代數位簽章標準之一。ML-DSA 基於模組格理論問題,具有高安全性與量子抗性,預期將成為嵌入式安全應用中重要的簽章技術。

ML-DSA 的演算法特性包含大量的模數多項式計算(透過 NTT 實現)、高頻率的隨機雜湊運算(透過 SHAKE/Keccak 實現)以及簽章中的重複性拒絕取樣邏輯(Fiat-Shamir with Aborts 結構),使其在硬體層面實作上面臨高效能與面積間的權衡挑戰。目前已有數篇研究文獻針對 ML-DSA 提出 RTL 或 FPGA 實作架構,然而多數方案為追求效能,採用了多組相同子模組(如多個 Keccak 核心與多組 NTT 單元)以達到併行處理,但也造成整體硬體面積大幅增加,難以部署於資源受限之嵌入式平台中。

本研究的核心目標即在於突破此瓶頸,設計出面積精簡且具高效能之硬體加速器。為此,我們提出一個創新的設計策略:
\begin{itemize}
    \item 僅使用一組 Keccak 雜湊模組與一組 NTT 運算模組;
    \item 以模組內部高平行度與運算排程最佳化方式,提升模組利用率;
    \item 實現具備低面積、高吞吐量、相容 AXI-4 介面的 ML-DSA 硬體加速器。
\end{itemize}

本設計特別針對 ML-DSA-44(對應 NIST 安全等級 Category 2)進行最佳化,並以 RTL/Verilog 撰寫實現,可部署於 FPGA 與 ASIC 平台中。實驗結果顯示,在硬體面積顯著小於現有文獻設計的條件下,仍能提供具有競爭力的簽章延遲與效能,充分展現本架構在實務應用中的潛力。


\section{章節安排}
本論文一共分為六章,其編排如下:
\begin{itemize}
\setlength{\itemsep}{2pt}
\setlength{\parsep}{2pt}
\setlength{\parskip}{2pt}
  \item 第1章緒論,說明研究動機、研究方向與論文架構。
  \item 第2章演算法與背景介紹:介紹後量子密碼學、模組格理論、ML-DSA 演算法結構與其運作流程。
  \item 第3章ML-DSA 模組分析與架構設計:探討數位簽章過程中所需的主要模組與運算瓶頸。
  \item 第4章相容 AXI-4 介面之硬體加速器設計:詳細說明所設計的 RTL 架構、模組分工與時序設計。
  \item 第5章實現與實驗結果分析:包含 FPGA 與 ASIC 實作結果、效能分析與比較。
  \item 第6章結論與未來展望:總結本研究成果並提出後續改進方向。
\end{itemize}

\ifdefined\integrated\else
\input{backpages/ntust_backpages.tex}
\end{document} \fi
